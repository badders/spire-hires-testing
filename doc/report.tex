\documentclass[a4paper]{article}

\usepackage[intlimits]{amsmath}
\usepackage{mathrsfs}
\usepackage{amsthm}
\usepackage{amssymb}
\usepackage{cancel}
\usepackage{graphicx}
\usepackage{color}
\usepackage{natbib}
\usepackage{hyperref}
\usepackage{listings}
\usepackage{parskip}
\usepackage[margin=1.3in]{geometry}
\usepackage{abstract}
\usepackage{multicol}
\usepackage{float}

\citestyle{apalike}
\bibliographystyle{apalike}

\providecommand{\e}[1]{\times10^{#1}}
\providecommand{\units}[1]{\;\mathrm{#1}}
\providecommand{\data}[2]{$#1\units{#2}$}
\providecommand{\diff}[2]{\frac{\partial #1}{\partial #2}}
\providecommand{\ddiff}[2]{\frac{\partial^2 #1}{\partial #2^2}}
\providecommand{\tdiff}[2]{\frac{\mathrm{d} #1}{\mathrm{d} #2}}
\providecommand{\infint}[2]{\int\limits_{-\infty}^{\infty}{#1}\;\mathrm{d}#2}
\providecommand{\dd}{\;\mathrm{d}}
\providecommand{\MAA}{\text{\AA}}
\providecommand{\abs}[1]{\left| #1 \right|} % for absolute value
\providecommand{\avg}[1]{\left< #1 \right>} % for average
\providecommand{\grad}[1]{\gv{\nabla} #1} % for gradient
\providecommand{\bigo}[1]{\mathcal{O}\left( #1 \right)}

\let\divsymb=\div
\renewcommand{\div}[1]{\gv{\nabla} \cdot #1} % for divergence
\providecommand{\curl}[1]{\gv{\nabla} \times #1} % for curl
\providecommand{\MAA}{\text{\AA}}
\providecommand{\figwidth}{.45\columnwidth}
\newcommand\numberthis{\addtocounter{equation}{1}\tag{\theequation}}
\numberwithin{equation}{section}

\renewcommand{\bibname}{References}

\title{{A study of the characteristics of HiRes output on Herschel SPIRE Observations}}
\author{Tom Badran}

\date{July to September 2015}

\begin{document}

\maketitle

\section{Introduction}

This work is a continuation of the work in my final year undergraduate project \cite{badran} in order to establish a set of criteria for deciding whether HiRes should be applied in the SPIRE data processing pipeline, and sensible values for some of the parameters. This paper is recommended reading in order to study the techniques used for analyzing the image fidelity improvement achieved through HiRes.

\section{HiRes time Complexity}

\begin{figure}[H]
    \centering
    \includegraphics[width=0.85\linewidth]{runtime.pdf}
    \caption{Measurements of HiRes runtime as a function of the input beam half-width pixel size}
    \label{hrtc}
\end{figure}

To measure the time complexity of HiRes, it was run with a set of beams cropped to have specific beam half-widths. These runs were timed and the results shown in figure \ref{hrtc} were obtained.

Clearly HiRes has a runtime complexity of approximately $\bigo{n^{1.8}}$, so runs in approximately quadratic time (as would be expected for convolution) as a function of beam width. Clearly for the shorter wavelength observations we want to be able to minimize this beam size as much as possible in order to be able to perform HiRes on as much of the data as possible in a reasonable time.

\section{Quality of HiRes as a function of input beam size}

To help make a decision on what beam size to use, we need to have an understanding of not just the time complexity, but also how the beam size affects the fidelity of the output from HiRes.

\begin{figure}[H]
    \centering
    \includegraphics[width=0.85\linewidth]{beam-size.pdf}
    \caption{Comparison of HiRes output to theoretical truth image, as a function of the input beam half-width pixel size.}
    \label{hrtc}
\end{figure}

\bibliography{references}

\end{document}
